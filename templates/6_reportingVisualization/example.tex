\documentclass[
  stu,
  floatsintext,
  longtable,
  nolmodern,
  notxfonts,
  notimes,
  colorlinks=true,linkcolor=blue,citecolor=blue,urlcolor=blue]{apa7}

\usepackage{amsmath}
\usepackage{amssymb}



\usepackage[bidi=default]{babel}
\babelprovide[main,import]{english}


% get rid of language-specific shorthands (see #6817):
\let\LanguageShortHands\languageshorthands
\def\languageshorthands#1{}

\RequirePackage{longtable}
\RequirePackage{threeparttablex}

\makeatletter
\renewcommand{\paragraph}{\@startsection{paragraph}{4}{\parindent}%
	{0\baselineskip \@plus 0.2ex \@minus 0.2ex}%
	{-.5em}%
	{\normalfont\normalsize\bfseries\typesectitle}}

\renewcommand{\subparagraph}[1]{\@startsection{subparagraph}{5}{0.5em}%
	{0\baselineskip \@plus 0.2ex \@minus 0.2ex}%
	{-\z@\relax}%
	{\normalfont\normalsize\bfseries\itshape\hspace{\parindent}{#1}\textit{\addperi}}{\relax}}
\makeatother




\usepackage{longtable, booktabs, multirow, multicol, colortbl, hhline, caption, array, float, xpatch}
\usepackage{subcaption}


\renewcommand\thesubfigure{\Alph{subfigure}}
\setcounter{topnumber}{2}
\setcounter{bottomnumber}{2}
\setcounter{totalnumber}{4}
\renewcommand{\topfraction}{0.85}
\renewcommand{\bottomfraction}{0.85}
\renewcommand{\textfraction}{0.15}
\renewcommand{\floatpagefraction}{0.7}

\usepackage{tcolorbox}
\tcbuselibrary{listings,theorems, breakable, skins}
\usepackage{fontawesome5}

\definecolor{quarto-callout-color}{HTML}{909090}
\definecolor{quarto-callout-note-color}{HTML}{0758E5}
\definecolor{quarto-callout-important-color}{HTML}{CC1914}
\definecolor{quarto-callout-warning-color}{HTML}{EB9113}
\definecolor{quarto-callout-tip-color}{HTML}{00A047}
\definecolor{quarto-callout-caution-color}{HTML}{FC5300}
\definecolor{quarto-callout-color-frame}{HTML}{ACACAC}
\definecolor{quarto-callout-note-color-frame}{HTML}{4582EC}
\definecolor{quarto-callout-important-color-frame}{HTML}{D9534F}
\definecolor{quarto-callout-warning-color-frame}{HTML}{F0AD4E}
\definecolor{quarto-callout-tip-color-frame}{HTML}{02B875}
\definecolor{quarto-callout-caution-color-frame}{HTML}{FD7E14}

%\newlength\Oldarrayrulewidth
%\newlength\Oldtabcolsep


\usepackage{hyperref}




\providecommand{\tightlist}{%
  \setlength{\itemsep}{0pt}\setlength{\parskip}{0pt}}
\usepackage{longtable,booktabs,array}
\usepackage{calc} % for calculating minipage widths
% Correct order of tables after \paragraph or \subparagraph
\usepackage{etoolbox}
\makeatletter
\patchcmd\longtable{\par}{\if@noskipsec\mbox{}\fi\par}{}{}
\makeatother
% Allow footnotes in longtable head/foot
\IfFileExists{footnotehyper.sty}{\usepackage{footnotehyper}}{\usepackage{footnote}}
\makesavenoteenv{longtable}

\usepackage{graphicx}
\makeatletter
\def\maxwidth{\ifdim\Gin@nat@width>\linewidth\linewidth\else\Gin@nat@width\fi}
\def\maxheight{\ifdim\Gin@nat@height>\textheight\textheight\else\Gin@nat@height\fi}
\makeatother
% Scale images if necessary, so that they will not overflow the page
% margins by default, and it is still possible to overwrite the defaults
% using explicit options in \includegraphics[width, height, ...]{}
\setkeys{Gin}{width=\maxwidth,height=\maxheight,keepaspectratio}
% Set default figure placement to htbp
\makeatletter
\def\fps@figure{htbp}
\makeatother


% definitions for citeproc citations
\NewDocumentCommand\citeproctext{}{}
\NewDocumentCommand\citeproc{mm}{%
  \begingroup\def\citeproctext{#2}\cite{#1}\endgroup}
\makeatletter
 % allow citations to break across lines
 \let\@cite@ofmt\@firstofone
 % avoid brackets around text for \cite:
 \def\@biblabel#1{}
 \def\@cite#1#2{{#1\if@tempswa , #2\fi}}
\makeatother
\newlength{\cslhangindent}
\setlength{\cslhangindent}{1.5em}
\newlength{\csllabelwidth}
\setlength{\csllabelwidth}{3em}
\newenvironment{CSLReferences}[2] % #1 hanging-indent, #2 entry-spacing
 {\begin{list}{}{%
  \setlength{\itemindent}{0pt}
  \setlength{\leftmargin}{0pt}
  \setlength{\parsep}{0pt}
  % turn on hanging indent if param 1 is 1
  \ifodd #1
   \setlength{\leftmargin}{\cslhangindent}
   \setlength{\itemindent}{-1\cslhangindent}
  \fi
  % set entry spacing
  \setlength{\itemsep}{#2\baselineskip}}}
 {\end{list}}
\usepackage{calc}
\newcommand{\CSLBlock}[1]{\hfill\break\parbox[t]{\linewidth}{\strut\ignorespaces#1\strut}}
\newcommand{\CSLLeftMargin}[1]{\parbox[t]{\csllabelwidth}{\strut#1\strut}}
\newcommand{\CSLRightInline}[1]{\parbox[t]{\linewidth - \csllabelwidth}{\strut#1\strut}}
\newcommand{\CSLIndent}[1]{\hspace{\cslhangindent}#1}





\usepackage{newtx}

\defaultfontfeatures{Scale=MatchLowercase}
\defaultfontfeatures[\rmfamily]{Ligatures=TeX,Scale=1}





\title{Using Quarto to Generate Documents in APA Style (7th Edition)}


\shorttitle{Template for the apaquarto Extension}


\usepackage{etoolbox}









\authorsnames[{1},{1},{2,3},{4}]{Ana Fulano,Blanca Zutano,Carina
Mengano,Dolorita C. Perengano}







\authorsaffiliations{
{Department of Psychology, Ana and Blanca's University},{Carina's
Primary Affiliation},{Carina's Secondary Affiliation},{Buffalo, NY }}




\leftheader{Fulano, Zutano, Mengano and C.}



\abstract{This document is a template demonstrating the apaquarto
format. }

\keywords{keyword1, keyword2, keyword3}

\authornote{\par{\addORCIDlink{Ana
Fulano}{0000-0000-0000-0001}}\par{\addORCIDlink{Blanca
Zutano}{0000-0000-0000-0002}}\par{\addORCIDlink{Carina
Mengano}{0000-0000-0000-0003}}\par{\addORCIDlink{Dolorita C.
Perengano}{0000-0000-0000-0004}} 
\par{Carina Mengano is now at Generic University. }
\par{   The authors have no conflicts of interest to disclose.    Author
roles were classified using the Contributor Role Taxonomy (CRediT;
\href{https://credit.niso.org}{credit.niso.org}) as follows:  Ana
Fulano:   conceptualization, writing; Blanca Zutano:   project
administration, formal analysis; Carina Mengano:   formal
analysis, writing; Dolorita C. Perengano:   writing, methodology, formal
analysis}
\par{Correspondence concerning this article should be addressed to Ana
Fulano, Department of Psychology, Ana and Blanca's University, 1234
Capital
St., Albany, NY 12084-1234, USA, Email: \href{mailto:sm@example.org}{sm@example.org}}
}

\makeatletter
\let\endoldlt\endlongtable
\def\endlongtable{
\hline
\endoldlt
}
\makeatother

\urlstyle{same}



\makeatletter
\@ifpackageloaded{caption}{}{\usepackage{caption}}
\AtBeginDocument{%
\ifdefined\contentsname
  \renewcommand*\contentsname{Table of contents}
\else
  \newcommand\contentsname{Table of contents}
\fi
\ifdefined\listfigurename
  \renewcommand*\listfigurename{List of Figures}
\else
  \newcommand\listfigurename{List of Figures}
\fi
\ifdefined\listtablename
  \renewcommand*\listtablename{List of Tables}
\else
  \newcommand\listtablename{List of Tables}
\fi
\ifdefined\figurename
  \renewcommand*\figurename{Figure}
\else
  \newcommand\figurename{Figure}
\fi
\ifdefined\tablename
  \renewcommand*\tablename{Table}
\else
  \newcommand\tablename{Table}
\fi
}
\@ifpackageloaded{float}{}{\usepackage{float}}
\floatstyle{ruled}
\@ifundefined{c@chapter}{\newfloat{codelisting}{h}{lop}}{\newfloat{codelisting}{h}{lop}[chapter]}
\floatname{codelisting}{Listing}
\newcommand*\listoflistings{\listof{codelisting}{List of Listings}}
\makeatother
\makeatletter
\makeatother
\makeatletter
\@ifpackageloaded{caption}{}{\usepackage{caption}}
\@ifpackageloaded{subcaption}{}{\usepackage{subcaption}}
\makeatother

% From https://tex.stackexchange.com/a/645996/211326
%%% apa7 doesn't want to add appendix section titles in the toc
%%% let's make it do it
\makeatletter
\xpatchcmd{\appendix}
  {\par}
  {\addcontentsline{toc}{section}{\@currentlabelname}\par}
  {}{}
\makeatother

%% Disable longtable counter
%% https://tex.stackexchange.com/a/248395/211326

\usepackage{etoolbox}

\makeatletter
\patchcmd{\LT@caption}
  {\bgroup}
  {\bgroup\global\LTpatch@captiontrue}
  {}{}
\patchcmd{\longtable}
  {\par}
  {\par\global\LTpatch@captionfalse}
  {}{}
\apptocmd{\endlongtable}
  {\ifLTpatch@caption\else\addtocounter{table}{-1}\fi}
  {}{}
\newif\ifLTpatch@caption
\makeatother

\begin{document}

\maketitle


\setcounter{secnumdepth}{-\maxdimen} % remove section numbering

\setlength\LTleft{0pt}


This is my introductory paragraph. The title will be placed above it
automatically. \emph{Do not start with an introductory heading} (e.g.,
``Introduction''). The title acts as your Level 1 heading for the
introduction.

Details about writing headings with markdown in APA style are
\href{https://wjschne.github.io/apaquarto/writing.html\#headings-in-apa-style}{here}.
In general, level 1 headings should be reserved for the Method, Results,
Discussion, References, and Appendices sections.

\subsection{Citations}\label{citations}

See
\href{https://quarto.org/docs/authoring/footnotes-and-citations.html}{here}
for instructions on setting up citations and references.

A parenthetical citation requires square brackets
(\citeproc{ref-CameronTrivedi2013}{Cameron \& Trivedi, 2013}). This
reference was in my bibliography file. An in-text citation is done like
so:

Cameron and Trivedi (\citeproc{ref-CameronTrivedi2013}{2013}) make some
important points \ldots{}

See
\href{https://wjschne.github.io/apaquarto/writing.html\#references}{here}
for explanations, examples, and citation features exclusive to
apaquarto. For example, apaquarto can automatically handle possessive
citations:

Schneider and McGrew's (\citeproc{ref-schneider2012cattell}{2012})
position was \ldots{}

\subsection{Masking Author Identity for Peer
Review}\label{masking-author-identity-for-peer-review}

Setting \texttt{mask} to \texttt{true} will remove author names,
affiliations, and correspondence from the title page. Any references
listed in the \texttt{masked-citations} field will be masked as well.
See
\href{https://wjschne.github.io/apaquarto/writing.html\#masked-citations-for-anonymous-peer-review}{here}
for more information.

\subsection{Block Quotes}\label{block-quotes}

Sometimes you want to give a longer quote that needs to go in its own
paragraph. Block quotes are on their own line starting with the
\textgreater{} character. For example, Austen's
(\citeproc{ref-austenMansfieldPark1990}{1814/1990}) \emph{Mansfield
Park} has some memorable insights about the mind:

\begin{quote}
If any one faculty of our nature may be called more wonderful than the
rest, I do think it is memory. There seems something more speakingly
incomprehensible in the powers, the failures, the inequalities of
memory, than in any other of our intelligences. The memory is sometimes
so retentive, so serviceable, so obedient; at others, so bewildered and
so weak; and at others again, so tyrannic, so beyond control! We are, to
be sure, a miracle every way; but our powers of recollecting and of
forgetting do seem peculiarly past finding out. (p.~163)
\end{quote}

\subsection{Math and Equations}\label{math-and-equations}

Inline math uses \LaTeX syntax with single dollar signs. For example,
the reliability coefficient of my measure is \(r_{XX}=.95\).

If you want to display and refer to a specific formula, enclose the
formula in two dollar signs. After the second pair of dollar signs,
place the label in curly braces. The label should have an \texttt{\#eq-}
prefix. To refer to the formula, use the same label but with the
\texttt{@} symbol. For example, Equation~\ref{eq-euler} is Euler's
Identity, which is much admired for its elegance.

\begin{equation}\phantomsection\label{eq-euler}{
e^{i\pi}+1=0
}\end{equation}

A more practical example is the z-score equation seen in
Equation~\ref{eq-zscore}.

\begin{equation}\phantomsection\label{eq-zscore}{
z=\frac{X-\mu}{\sigma}
}\end{equation}

If no identifier label is given, a centered equation in display mode
will have no identifying number:

\[
\sigma_e=\sigma_y\sqrt{1-r_{xy}^2}
\]

\subsection{Displaying Figures}\label{displaying-figures}

Do you want the tables and figures to be at the end of the document? You
can set the \texttt{floatsintext} option to false. The reference labels
will work no matter where they are in the text.

A reference label for a figure must have the prefix \texttt{fig-}, and
in a code chunk, the caption must be set with \texttt{fig-cap}. Captions
are in
\href{https://apastyle.apa.org/style-grammar-guidelines/capitalization/title-case}{title
case}.

\begin{figure}[!htbp]

{\caption{{The Figure Caption}{\label{fig-myplot}}}}

\includegraphics{example_files/figure-pdf/fig-myplot-1.pdf}

{\noindent \emph{Note.} This is the note below the figure.}

\end{figure}

To refer to any figure or table, use the \texttt{@} symbol followed by
the reference label (e.g., Figure~\ref{fig-myplot}).

\subsection{Displaying Tables}\label{displaying-tables}

We can make a table the same way as a figure. Generating a table that
conforms to APA format in all document formats can be tricky. When the
table is simple, the \texttt{kable} function from knitr works well. Feel
free to experiment with different methods, but I have found that David
Gohel's \href{https://davidgohel.github.io/flextable/}{flextable} to be
the best option when I need something more complex.

\begin{table}

{\caption{{The Table Caption.}{\label{tbl-mytable}}}}

\begin{longtable}[]{@{}rl@{}}
\toprule\noalign{}
Numbers & Letters \\
\midrule\noalign{}
\endhead
\bottomrule\noalign{}
\endlastfoot
1 & A \\
2 & B \\
3 & C \\
4 & D \\
\end{longtable}

{\noindent \emph{Note.} The note below the table.}

\end{table}

To refer to this table in text, use the \texttt{@} symbol followed by
the reference label like so: As seen in Table~\ref{tbl-mytable}, the
first few numbers and letters of the alphabet are displayed.

\subsection{Tables and Figures Spanning Two Columns in Journal
Mode}\label{tables-and-figures-spanning-two-columns-in-journal-mode}

When creating tables and figures in journal mode, care must be taken not
to make figures and tables wider than the columns, otherwise \LaTeX
sometimes makes them disappear.

As demonstrated in Figure~\ref{fig-twocolumn}, you can make figures
tables span the two columns by setting the \texttt{apa-twocolumn} chunk
option to \texttt{true}.

\begin{figure*}[tp]

{\caption{{A Figure Spanning Two Columns When in Journal
Mode}{\label{fig-twocolumn}}}}

\begin{center}
\includegraphics{example_files/figure-pdf/fig-twocolumn-1.pdf}
\end{center}

{\noindent \emph{Note.} Figures in two-column mode are only different
for jou mode in .pdf documents}

\end{figure*}

\subsection{Footnotes}\label{footnotes}

A footnote is usually displayed at the bottom of the page on which the
footnote occurs. A short note can be specified with the
\texttt{\^{}{[}My\ note\ here{]}} syntax.\footnote{Here is my short
  footnote!} A longer note can be specified with the
\texttt{{[}\^{}id{]}} syntax with the text specified on a separate line
like so \texttt{{[}\^{}id{]}:\ Text\ here}.\footnote{This is a longer
  footnote. If it has multiple paragraphs, subsequent paragraphs need to
  be indented with two tabs.

  This paragraph is still part of the footnote because it is indented
  with two tabs.}

A regular paragraph without any indentation is not part of the footnote
and will be part of the main body of the document.

\subsection{Hypotheses, Aims, and
Objectives}\label{hypotheses-aims-and-objectives}

The last paragraph of the introduction usually states the specific
hypotheses of the study, often in a way that links them to the research
design.

\section{Method}\label{method}

General remarks on method. This paragraph is optional.

Not all papers require each of these sections. Edit them as needed.
Consult the \href{https://apastyle.apa.org/jars}{Journal Article
Reporting Standards} for what is needed for your type of article.

\subsection{Participants}\label{participants}

Who are they? How were they recruited? Report criteria for participant
inclusion and exclusion. Perhaps some basic demographic stats are in
order. A table is a great way to avoid repetition in statistical
reporting.

\subsection{Measures}\label{measures}

This section can also be titled \textbf{Materials} or
\textbf{Apparatus}. Whatever tools, equipment, or measurement devices
used in the study should be described.

\subsubsection{Measure A}\label{measure-a}

Describe Measure A.

\subsubsection{Measure B}\label{measure-b}

Describe Measure B.

\paragraph{Subscale B1.}\label{subscale-b1}

A paragraph after a 4th-level header will appear on the same line as the
header.

\paragraph{Subscale B2.}\label{subscale-b2}

A paragraph after a 4th-level header will appear on the same line as the
header.

\subparagraph{Subscale B2a.}\label{subscale-b2a}

A paragraph after a 5th-level header will appear on the same line as the
header.

\subparagraph{Subscale B2b.}\label{subscale-b2b}

A paragraph after a 5th-level header will appear on the same line as the
header.

\subsection{Procedure}\label{procedure}

What did participants do? How are the data going to be analyzed?

\section{Results}\label{results}

\subsection{Descriptive Statistics}\label{descriptive-statistics}

Describe the basic characteristics of the primary variables. My ideal is
to describe the variables well enough that someone conducting a
meta-analysis can include the study without needing to ask for
additional information.

Table~\ref{tbl-mymarkdowntable2} is an example of a plain markdown
table. Note the that the caption begins with a colon.

\begin{table}

{\caption{{My Caption.}{\label{tbl-mymarkdowntable2}}}}

\begin{longtable}[]{@{}cc@{}}
\toprule\noalign{}
Letters & Numbers \\
\midrule\noalign{}
\endhead
\bottomrule\noalign{}
\endlastfoot
A & 1 \\
B & 2 \\
C & 3 \\
\end{longtable}

{\noindent \emph{Note.} My note}

\end{table}

\section{Discussion}\label{discussion}

Describe results in non-statistical terms.

\subsection{Limitations and Future
Directions}\label{limitations-and-future-directions}

Every study has limitations. Based on this study, some additional steps
might include\ldots{}

\subsection{Conclusion}\label{conclusion}

Describe the main point of the paper.

\section{References}\label{references}

\phantomsection\label{refs}
\begin{CSLReferences}{1}{0}
\bibitem[\citeproctext]{ref-austenMansfieldPark1990}
Austen, J. (1990). \emph{Mansfield {P}ark}. Oxford University Press.
(Original work published 1814)

\bibitem[\citeproctext]{ref-CameronTrivedi2013}
Cameron, A. C., \& Trivedi, P. K. (2013). \emph{Regression analysis of
count data} (2nd ed.). Cambridge University Press.
\url{https://doi.org/10.1017/CBO9781139013567}

\bibitem[\citeproctext]{ref-schneider2012cattell}
Schneider, W. J., \& McGrew, K. S. (2012). The {Cattell-Horn-Carroll}
model of intelligence. In D. P. Flanagan \& P. L. Harrison (Eds.),
\emph{Contemporary intellectual assessment: {Theories}, tests, and
issues} (3rd ed., pp. 99--144). Guilford Press.
\url{https://psycnet.apa.org/record/2012-09043-004}

\end{CSLReferences}

\appendix

\section{My Appendix Title}\label{apx-a}

Appendices are created as level 1 headings with an identifier with an
\texttt{\#apx-} prefix. Appendix titles should be in title case and
should describe the content of the appendix.

If there is only one appendix, the label automatically inserted above
the the appendix title will be \textbf{Appendix}. If there are multiple
appendices, the labels \textbf{Appendix A}, \textbf{Appendix B},
\textbf{Appendix C} and so forth will be inserted above the titles.

To cite an appendix as a whole, reference it with the \texttt{@apx-}
prefix. For example, see \hyperref[apx-a]{Appendix~A} and
\hyperref[apx-b]{Appendix~B}.

This is an appendix with a table using markdown (see
Table~\ref{tbl-letters}).

\begin{table}

{\caption{{My Caption}{\label{tbl-letters}}}}

\begin{longtable}[]{@{}lll@{}}
\toprule\noalign{}
Col 1 & Col 2 & Col 3 \\
\midrule\noalign{}
\endhead
\bottomrule\noalign{}
\endlastfoot
A & B & C \\
E & F & G \\
A & G & G \\
\end{longtable}

{\noindent \emph{Note.} These are letters.}

\end{table}

\section{Another Appendix}\label{apx-b}

See Figure~\ref{fig-appendfig}, an example of an imported graphic using
markdown syntax.

\begin{figure}

{\caption{{Appendix Figure}{\label{fig-appendfig}}}}

\includegraphics{sampleimage.png}

{\noindent \emph{Note.} A \emph{note} below the figure}

\end{figure}






\end{document}
